\documentclass{scrartcl}
\usepackage[utf8]{inputenc}
\usepackage[ngerman]{babel}
\usepackage{booktabs}

\title{Projekt H.E.L.I.X.}
\author{ Fischer, Bastian
		\and
		Haß, Jonne
		\and
		Held, Benjamin
		\and
		Pump, Richard}
\date{}
\begin{document}
\maketitle
\section{Kurzbeschreibung}
\emph{Aus der Projektbeschreibung:}

Ein Selbst erstelltes Raumschiff im Weltraum, umgeben von 
langsam rotierenden Asteroiden. 
Dazu: Ein bunter Nebel, sowie eine Skybox mit zwei verschieden 
farbigen Sonnen.
Es gibt eine Außenperspektive sowie eine "Cockpit-Ansicht".
Interaktion bewegt das Raumschiff.

Compilieren am Besten per Build-Skript:
\emph{./build.sh project_helix}

\section{Tastaturbelegung}

\begin{tabular}{cl}
Taste: & Aktion \\
1-3 & Sekundäre Kamera umschalten. \\
a, d & Nach links oder rechts gieren. \\
w, s & Steigen oder Sinken. \\
q, s & Rollen nach links oder rechst.\\
f & Rakete abfeuern.\\
Umschalt/Shift (halten) & Nachbrenner aktivieren. \\
\end{tabular}

\section{Interessante Dinge}
\subsection{Rotationskörper}
\begin{itemize}
\item Rakete
\item Asteroid und Nebel
\item Sonnen und Planet
\item Planetenring
\end{itemize}

\section{Besondere Effekte}
\begin{itemize}
\item Blending auf Planetenring
\end{itemize}

\section{Animationen}
\begin{itemize}
\item Planeten und Asteroid drehen sich.
\item Schiff besitzt 6 Freiheitsgrade.
\item Rakete bewegt sich auf geradem Pfad und verschwindet dann.
\item "Reaper" bewegt sich auf Station zu.
\item Station bewegt sich um Planeten.
\end{itemize}

\subsection{Weiteres}
\begin{itemize}
\item "Reaper", Schiff und Station aus Blender.
\item Eindimensionale Textur auf Planetenring.
\item Partikelsysteme an Schiff und Rakete.
\end{itemize}
\end{document}