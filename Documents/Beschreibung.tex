\documentclass{scrartcl}
\usepackage[utf8]{inputenc}
\usepackage[ngerman]{babel}
\usepackage{booktabs}
\usepackage[pdfborder={0 0 0}, colorlinks=true, urlcolor=blue]{hyperref}

\title{Projekt H.E.L.I.X.}
\author{ Fischer, Bastian
		\and
		Haß, Jonne
		\and
		Held, Benjamin
		\and
		Pump, Richard}
\date{}
\begin{document}
\maketitle
\section{Kurzbeschreibung}
\emph{Aus der Projektbeschreibung:}

Ein Selbst erstelltes Raumschiff im Weltraum, umgeben von 
langsam rotierenden Asteroiden. 
Dazu: Ein bunter Nebel, sowie eine Skybox mit zwei verschieden 
farbigen Sonnen.
Es gibt eine Außenperspektive sowie eine "Cockpit-Ansicht".
Interaktion bewegt das Raumschiff.

Compilieren am Besten per Build-Skript:
\emph{./build.sh project\_helix}

\section{Tastaturbelegung}

\begin{tabular}{cl}
Taste: & Aktion \\
v & Primäre Kamera umschalten.\\
1-3 & Sekundäre Kamera umschalten. \\
A, D & Nach links oder rechts gieren. \\
W, S & Steigen oder Sinken. \\
Q, S & Rollen nach links oder rechts.\\
F & Rakete abfeuern.\\
V & Kameraperspektive ändern. \\
Umschalt/Shift (halten) & Nachbrenner aktivieren. \\
\end{tabular}

\section{Interessante Dinge}
\subsection{Objekte}
\begin{itemize}
\item Körper der Rakete ist durch den Rotationskörper realisiert.
\item Weltall-Nebel mit Hilfe von Partikeln.
\item Sonnen, Planeten und Asteroiden auf Grundlage einer selbst erstellten Kugel.
\item Sonnen dienen als punktförmige Lichtquelle.
\item Planetenring auf Basis eines selbst erstellten Torus'.
\end{itemize}

\subsection{Animationen}
\begin{itemize}
\item Planet und einige Asteroiden innerhalb des Feldes drehen sich.
\item Schiff besitzt 6 Freiheitsgrade.
\item Rakete bewegt sich auf geradem Pfad und verschwindet dann.
\item "Reaper"-Raumschiff bewegt sich auf Station zu, führt eine Animation aus und entfernt sich dann wieder. (Animationsdauer: einige Minuten)
\item Ein kleiner, reflektierender (zu Beginn vielleicht etwas schwer zu sehen) Monolith fliegt von den Asteroiden aus auf den Planeten zu.
\item Komet fliegt durch die Szene und dreht sich.
\end{itemize}

\subsection{Weiteres}
\begin{itemize}
\item "Reaper", zu steuerndes Raumschiff und Station aus Blender.
\item Eindimensionale Textur auf Planetenring, die reflektiert und transparente Ringsegmente enthält.
\item Partikelsysteme an Schiff, Komet und Rakete.
\end{itemize}

\section{Third-party Bibliotheken}
\begin{itemize}
\item Normals aus osgToy (\url{http://osgtoy.sourceforge.net}), nur zum Entwickeln, nicht im Hauptprogramm benötigt. Wird in \texttt{3rdparty/} mitgeliefert.
\end{itemize}

\end{document}
